% ATENÇÃO - veja com o seu orientador se você vai ter este capítulo e se este vai ter nome!
\chapter{Trabalhos Relacionados}
\label{cap:trabalhos:relacionados}


\section{Reconhecimento de acordes: uma abordagem via Multilayer Perceptron}
\label{cap:trabalhos:sec:relacionados:uso:citacoes}


De acordo com o artigo de \cite{cletoa2010reconhecimento} é possível reconhecer acordes utilizando uma arquitetura de rede neural tradicional, o perceptron multicamadas (Multilayer Perceptron - MLP).  

O reconhecimento de acordes pode ser modelado para ser usado em uma rede neural MLP aplicando como entrada as amplitudes obtidas pela transformada de Fourier e como saídas os valores que representam os acordes.  

Para chegar nos resultados esperados do trabalho foram usados como fonte de entrada os istrumentos : Piano, Cravo, Orgão e Violão. 

No treinamento foram usados 3 variações de acordes (formas diferentes de se tocar o mesmo acorde) para cada um dos 12 acordes da Tabela de acordes, totalizando 36 combinações, essas combinações foram gravadas pelos instrumetos escolhidos, totalizando 144 amostras.

O espectro de frequências foi dividido em 61 faixas (61 teclas do piano), a frequência de maior amplitude de cada faixa, obtida por meio da transforma de Fourier, foi coletada e utilizada como variável de entrada no MLP configurado com 61 nós na sua camada oculta.

Na validação foram gravadas 90 novas amostras de acordes com as seguintes caracteristicas: 

    Teste 1: Cada um dos 12 acordes tocados com o mesmo timbre (Piano);
    Teste 2: O acorde C com 11 timbres não usados no treinamento;
    Teste 3: Três variações do acorde C com 4 timbres semelhantes aos treinados;
    Teste 4: Os 12 acordes com cada um dos 4 timbres usados no treinamento;
    Teste 5: Sete acordes tocados em um violão real.

Observou-se que o treinamento da rede realizado com as 144 amostras de treinamento obtidas convergiu por volta da iteração 300 e que o erro quadrático médio, após 1000 iterações, atingiu um valor de 0,00061.

Os resultados obtidos revelam que essa metodologia pode ser usada com sucesso, proporcionando taxas de reconhecimento médio de 84. Dessa forma, apesar da grande quantidade de limitações impostas, dentre elas o baixo número de acordes (12) e instrumentos (4) considerados e o baixo número de amostras utilizadas no treinamento (144), podemos considerar que os resultados foram promissores.



%OBS: colocar mais um trabalho relacionado 

% Apresente aqui os trabalhos similares ao seu trabalho ou que são importantes para o entendimento do seu trabalho...

% (ATENÇÃO - veja com o seu orientador se você vai ter este capítulo e se este vai ter nome!)


% Este é um exemplo do uso de citações no texto \cite{}.

% Segundo \citeonline[p.~56]{Moore:2000:CMC:333067.333074} para citações textuais...

% De acordo com o trabalho de \citeonline{Moore:2000:CMC:333067.333074} para citações textuais não tão específicas...


%---------------------------------------------------%
% \section{Considerações Finais}
% \label{cap:trabalhos:relacionados:sec:consideracoes:finais}

% Esta é uma sugestão de seção para dar um fechamento em cada uma dos capítulos.

% (ATENÇÃO - veja com o seu orientador se é uma seção necessária (pois trate-se de estilo de escrita))